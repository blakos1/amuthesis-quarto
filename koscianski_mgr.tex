% This is a LaTeX thesis template for Adam Mickiewicz University.
% to be used with Rmarkdown
% This template was produced by Jakub Nowosad
% Version: 16 February 2020

% Inspired by:
% This is a LaTeX thesis template for Monash University.
% to be used with Rmarkdown
% This template was produced by Rob Hyndman
% Version: 6 September 2016

\documentclass{amuthesis}
% \usepackage[polish]{babel}
\usepackage{polski}
\renewcommand{\figurename}{Rycina} % Redefine default figure caption %
\renewcommand{\tablename}{Tabela} % Redefine default table caption %
%%%%%%%%%%%%%%%%%%%%%%%%%%%%%%%%%%%%%%%%%%%%%%%%%%%%%%%%%%%%%%%
% Add any LaTeX packages and other preamble here if required
%%%%%%%%%%%%%%%%%%%%%%%%%%%%%%%%%%%%%%%%%%%%%%%%%%%%%%%%%%%%%%%
\usepackage{booktabs,tabularx} % Allows kableExtra to work %
\usepackage{indentfirst} % Adds indent in the first paragraph %
\usepackage{bookmark} % Adds indent in the first paragraph %

\author{Błażej Kościański}
\title{Porównanie metod określania zmian struktury przestrzennej
kategorii pokrycia terenu}
\def\titleeng{TODO}
\def\degreetitle{Praca magisterska}
\def\major{Geoinformacja}
\def\albumid{444861}
\def\thesisyear{2023}

% Add subject and keywords below
\hypersetup{
     %pdfsubject={The Subject},
     %pdfkeywords={Some Keywords},
     pdfauthor={Błażej Kościański},
     pdftitle={Porównanie metod określania zmian struktury przestrzennej
kategorii pokrycia terenu},
     pdfproducer={quarto with LaTeX}
}

\bibliography{thesis.bibpackages.bib}

\begin{document}

\pagenumbering{arabic}

\titlepage

\bookmarksetup{startatroot}

\hypertarget{streszczenie}{%
\chapter*{Streszczenie}\label{streszczenie}}
\addcontentsline{toc}{chapter}{Streszczenie}

\markboth{Streszczenie}{Streszczenie}

\textbf{Abstrakt}

Streszczenie powinno przedstawiać skrótowo główny problem pracy i jego
rozwiązanie. Możliwa struktura streszczenia to: (1) 1-3 zdania wstępu do
problemu (czym się zajmujemy, dlaczego jest to ważne, jakie są
problemy/luki do wypełnienia), (2) 1 zdanie opisujące cel pracy, (3) 1-3
zdania przedstawiające użyte materiały (dane) i metody (techniki,
narzędzia), (4) 1-3 zdania obrazujące główne wyniki pracy, (5) 1-2
zdania podsumowujące; możliwe jest też określenie dalszych
kroków/planów.

Słowa kluczowe: (4-6 słów/zwrotów opisujących treść pracy, które nie
wystąpiły w tytule)

\textbf{Abstract}

The abstract must be consistent with the above text.

Keywords: (as stated before)

\newpage

\setstretch{1.2}\sf\tighttoc\doublespacing

\bookmarksetup{startatroot}

\hypertarget{sec-wprowadzenie}{%
\chapter{Wprowadzenie}\label{sec-wprowadzenie}}

Informacje geograficzne stanowią wyniki selekcji i przetwarzania danych
dotyczących aspektów otaczającej nas przestrzeni geograficznej.
Pozwalają na bardziej zrozumiałe i efektywne analizowanie,
interpretowanie oraz modelowanie złożonych zjawisk i procesów
zachodzących w naszym otoczeniu. Informacje geograficzne i ich aspekty
nie stanowią niepodważalnych faktów, lecz często powstają w wyniku
działań jednostek, jak i wspólnych wysiłków grup ekspertów, którzy
zajmują się wyborem, analizą i klasyfikacją danych geograficznych
\autocite{WhatIsLandCover}. W procesie tworzenia informacji
geograficznych istnieje zatem pewien stopień subiektywności, który może
wpłynąć na ostateczny kształt (ostateczną postać?) tych informacji, ich
interpretację, jak i na ich użyteczność w kontekście innych zastosowań.
Przykładem informacji geograficznej, której ostateczna postać zależna
jest od założeń przyjętych w trakcie tworzenia danych przestrzennych na
ich podstawie, jest pokrycie terenu.

Przyjmuje się, że termin pokrycie terenu obejmuje zbiór wszelkich
elementów obecnych na powierzchni Ziemi. W elementy pokrycia terenu
włączają się obiekty związane z działalnością człowieka, skutkami sił
przyrody oraz wszelkie inne istniejące obiekty, które mogą znaleźć się w
przestrzeni geograficznej. Tworzenie dokładnych i wiarygodnych danych
dotyczących pokrycia terenu jest niezbędne w kontekście wielu
zastosowań, takich jak planowanie przestrzenne, ochrona środowiska, czy
analiza zmian klimatycznych. Ostateczna forma tych danych jest jednak w
dużej mierze determinowana przez wybory i założenia dokonywane w
procesie ich tworzenia. W tym kontekście, analiza pokrycia terenu staje
się istotnym polem badań, które skupia się na zarówno na technicznych
aspektach zbierania danych, jak i na ich semantycznej interpretacji.

Dane oraz wynikowe mapy pokrycia terenu są rezultatem skomplikowanego
procesu przetwarzania i analizy danych przestrzennych najczęściej w
postaci obrazów satelitarnych. Na początku tego procesu, satelity
wyposażone w sensory rejestrują obrazy Ziemi z różnych zakresów
widmowych. Uzyskane obrazy mogą być interpretowane manualnie przez grupy
specjalistów. Pozwala to na uzyskanie map pokrycia terenu o wysokiej
dokładności, kosztem długiego procesu ich tworzenia. Dużo mniej
czasochłonną metodą jest przetwarzanie przy użyciu algorytmów.
Umożliwiają one szybką, automatyczną identyfikację i klasyfikację
różnych typów powierzchni kosztem mniejszej dokładności mapy wynikowej.
Ostatecznie, dane przekształcone w mapy pokrycia terenu mogą posłużyć do
analiz zmian pokrycia terenu.

Celem analiz zmian pokrycia terenu jest przede wszystkim monitorowanie i
pogłębienie aktualnej wiedzy na temat ewolucji otaczającego nas
krajobrazu. Jest to istotne w kontekście ochrony przyrody, planowania
przestrzennego, oceny wpływu inwestycji i infrastruktury na środowisko,
a także w badaniach naukowych dotyczących zmian klimatycznych,
bioróżnorodności oraz innych procesów ekologicznych
\autocite{ChangeDetectionTechniques}. Dzięki analizie zmian pokrycia
terenu można identyfikować obszary zagrożone degradacją, monitorować
skutki urbanizacji, deforestacji czy erozji, co umożliwia podejmowanie
odpowiednich działań w celu zrównoważonego zarządzania środowiskiem i
zachowaniem jego integralności.

W badaniach nad zmianami pokrycia terenu wykorzystuje się różnorodne
metody analityczne \autocite{ChangeDetectionTechniques}. Niemniej
jednak, wiele z tych technik koncentruje się na analizie zmian na
poziomie indywidualnych komórek w siatce rastra. Choć podejście to może
dostarczać użytecznych informacji dotyczących trendów zmian pokrycia
terenu na niewielkich obszarach, charakteryzuje się istotnymi
ograniczeniami w kontekście interpretacji wyników. Szczególnie w
przypadku badań obejmujących rozległe terytoria, takie jak kraje czy
nawet kontynenty, bardziej efektywne staje się zastosowanie metod
opartych na analizie struktur przestrzennych \textcite{Netzel2015}.
Głównym założeniem tych metod jest przekształcenie danych z postaci
pojedynczych wartości komórek rastra w sygnatury przestrzenne.

Sygnatury przestrzenne stanowią statystyczny opis układów przestrzennych
kategorii pokrycia terenu na mniejszych, wydzielonych obszarach w
obrębie całego zbioru danych. W celu porównania ze sobą dwóch sygnatur
przestrzennych, wykorzystywane są miary niepodobieństwa. Umożliwiają one
określenie w jakim stopniu dwa analizowane obszary się od siebie różnią.
Opracowane zostało wiele różnych miar niepodobieństwa, takich jak
odległość euklidesowa, odległość Canberra, metryka Wave Hedges,
współczynnik podobieństwa Jaccarda, odległość Jensena-Shannona czy
dywergencja Pearsona \autocite{Cha2007}. Współcześnie jednak nie
określono, która z tych miar jest najbardziej zgodna zarówno z
postrzeganiem przez człowieka, jak i wpływem zmian na procesy
środowiskowe.

Celem tej pracy było porównanie metod określania zmian struktury
przestrzennej kategorii pokrycia terenu w kontekście ich korelacji z
postrzeganiem zmian przestrzennych przez człowieka. W celu realizacji
tego zadania przeprowadzona została ankieta, w której zadaniem
respondentów było określenie stopnia podobieństwa między parami rastrów.
Badania przeprowadzone zostały na rastrach składających się wyłącznie z
dwóch lub trzech kategorii. Wyniki ankiety zestawione zostały z
wartościami 46 miar niepodobieństwa. Na tej podstawie, do dalszej
analizy wybrane zostało 8 miar niepodobieństwa charakteryzujących się
największą zgodnością z ludzką percepcją zmian przestrzennych.

\bookmarksetup{startatroot}

\hypertarget{sec-lit}{%
\chapter{Przegląd literatury}\label{sec-lit}}

\textcolor{red}{
cały rozdział o tym jakie są rodziny miar odległości, skąd się wywodzą, kto je stworzył i jak są liczone
* dodać rozdział 2 przegląd literatury - rozdział teoretyczny, opisujący czym są kompozycja, konfiguracja, macierze współwystępowania, wektor współwystępowania, metryki entropia, informacja wzajemna, na czym polega porównywanie dwóch obrazów i miary niepodobieństwa i przykłady.
}

\bookmarksetup{startatroot}

\hypertarget{sec-materialy}{%
\chapter{Materiały}\label{sec-materialy}}

\textcolor{red}{
* każdy rozdział zacząć opisując co w ogóle będzie w tym rozdziale
* tabele z parametrami fract dim i proporcjami udziału kategorii
* wykres z entropią i mutinf, relmutinf udowadniający że obliczenia zostały jakoś zwalidowane
* jakoś zaznaczyć albo ponumerować rastry wykorzystane w ankiecie
* dodać podrozdział dotyczący obliczonych metryk
* przeliczyć wszystkie metryki z philentropy dla wszystkich par rastrów, policzyć między miarami korelacje, sprawdzić które miary się grupują grupowanie hierarchiczne, hierarchical cluster/ hclust
}

\bookmarksetup{startatroot}

\hypertarget{sec-metody}{%
\chapter{Metody}\label{sec-metody}}

\hypertarget{sec-przyg-danych}{%
\section{Przygotowanie danych}\label{sec-przyg-danych}}

Zbiór rastrów został przygotowany w oparciu o wykorzystanie funkcji
nlm\_fbm z pakietu NLMR. Funkcja ta bazuje na symulowaniu ułamkowych
ruchów Browna. \textless tu opisać jak to działa?\textgreater{}
Najważniejszym założeniem przy tworzeniu zbioru obrazów było
przygotowanie ich w sposób umożliwiający uzyskanie pełnej reprezentacji
wszystkich możliwych wartości przestrzennej kompozycji jak i
konfiguracji.

W kolejnym etapie obliczone zostały wybrane metryki: entropia (ent),
informacja wzajemna (mutinf) oraz względna informacja wzajemna
(relmutinf), które pozwoliły na potwierdzenie uzyskania oczekiwanego
rozkładu kompozycji i konfiguracji wewnątrz zbioru rastrów. W celu
stwierdzenia zależności między liczbą kategorii na rastrach a
postrzeganiem, przez ankietowanych, podobieństw w strukturach
przestrzennych, wygenerowane zostały zarówno zbiory rastrów składających
się z dwóch kategorii, jak i trzech kategorii. Przykład jednego ze
zbiorów wygenerowanych rastrów przedstawia rycina 1.

Ryc. 1. Przykład zbioru wygenerowanych rastrów (2 kategorie pokrycia
terenu).

W kolejnym etapie, rastry z każdego zbioru zostały połączone w pary. W
ten sposób, dla każdego ze zbiorów, otrzymany został zbiór wszystkich
możliwych par rastrów. Pozwoliło to na wybranie podzbioru par rastrów,
które umieszczone zostały w ankiecie.

\hypertarget{cel-badania}{%
\section{Cel badania}\label{cel-badania}}

Celem pierwszej ankiety była wstępna analiza zależności wśród metod
określania zmian struktury przestrzennej kategorii pokrycia terenu.
Przeprowadzenie ankiety pozwoliło także na wyznaczenie dalszego kierunku
badań, jak i celów, które miałyby zostać osiągnięte w wyniku kolejnej
ankiety.

\hypertarget{pruxf3ba-badawcza}{%
\section{Próba badawcza}\label{pruxf3ba-badawcza}}

Pierwsza ankieta przeprowadzona została w listopadzie 2022 roku. W
badaniach łącznie udział wzięło 50 studentów Wydziału Nauk
Geograficznych i Geologicznych im. Adama Mickiewicza w Poznaniu. Pytania
o odnoszące się do przynależności demograficzno-społecznych
ankietowanych zostały celowo pominięte zarówno w celu pełnej
anonimizacji udzielonych odpowiedzi oraz aby ankietowani czuli się
swobodniej podczas wypełniania formularza.

\hypertarget{forma-ankiety}{%
\section{Forma ankiety}\label{forma-ankiety}}

Ze względu na charakterystykę tematu badań ankieta przeprowadzona
została w sposób cyfrowy. Ankieta stworzona została w formie aplikacji
internetowej za pomocą języka programowania R, na podstawie pakietów
shiny oraz shinysurveys. Sama aplikacja umieszczona została na
platformie shinyapps.io (https://www.shinyapps.io/), w celu jak
największego ułatwienia ankietowanym możliwości wypełnienia formularza.

\hypertarget{struktura-pytaux144}{%
\subsection{Struktura pytań}\label{struktura-pytaux144}}

Przy każdym pytaniu zadaniem ankietowanych było określenie podobieństwa
między dwoma obrazami. Sposób przygotowania zbioru par wygenerowanych
obrazów opisany został w podrozdziale \#sec-przyg-danych. Do określenia
podobieństwa ankietowani mieli do dyspozycji pięciostopniową skalę
Likerta. Odpowiedzi uwzględniały pełen teoretyczny zakres podobieństwa
podzielony na pięć przedziałów: brak podobieństwa, bardzo małe
podobieństwo, umiarkowane podobieństwo, bardzo duże podobieństwo oraz
pełne podobieństwo. Zastosowanie skali Likerta o nieparzystej liczbie
przedziałów pozwoliło na zastosowanie przedziału środkowego, którego
celem było reprezentowanie odpowiedzi neutralnych lub trudnych do
określenia. Początkowo, zamiast skali Likerta planowano wykorzystać
skalę liczbową, w zakresie mieszczącym się od 1 do 100, jednakże
zrezygnowano z tego pomysłu, jako że znaczenie wartości na skali
liczbowej może być interpretowane inaczej przez każdego respondenta oraz
skala ta nie pozwala na uwzględnienie wspomnianej wcześniej odpowiedzi
neutralnej. Należy także wspomnieć, że przed przystąpieniem do
wypełnienia ankiety respondenci nie zostali poinformowani o możliwych
interpretacjach skali podobieństwa między obrazami, a jedynie o tym, w
jaki sposób pary obrazów mogą się od siebie różnić (przestrzenna
kompozycja i konfiguracja). Przykład pytania uwzględnionego w pierwszej
wersji ankiety przedstawia Rycina 1.

Ryc. 1. Przykład pytania uwzględnionego w pierwszej wersji ankiety (2
kategorie).

Pytania przedstawione ankietowanym zostały ułożone w uprzednio
określonej kolejności. Każdy formularz składał się z 48 pytań, przy czym
zostały one podzielone na dwie grupy po 24 pytania. Pierwsza połowa
pytań składała się z par obrazów uwzględniających wyłącznie dwie
kategorie pokrycia terenu, natomiast druga połowa pytań z par obrazów
uwzględniających trzy kategorie pokrycia terenu. Dla obu grup pytań
losowo wybrane zostało 6 par obrazów różniących się wyłącznie
konfiguracją, kolejne 6 par wyłącznie kompozycją, a następnie pozostałe
12 różniących się zarówno konfiguracją jak i kompozycją. Taki sposób
losowania pytań pozwolił na zredukowanie liczby odpowiedzi wymaganych od
respondentów, jak i ograniczenie wpływu błędu selekcji, który powstałby
w wyniku niewłaściwej próby danych do badania.

\bookmarksetup{startatroot}

\hypertarget{sec-wyniki}{%
\chapter{Wyniki}\label{sec-wyniki}}

Każdy respondent odpowiedział na 48 pytań, co oznacza, że uzyskano
łącznie 2400 odpowiedzi na wszystkie pytania.

\begin{itemize}
\tightlist
\item
  spośród miar niepodobieństwa uwzględnionych w analizie, miara
  Wave-Hedges charakteryzuje się największą korelacją z odpowiedziami
  ankietowanych.
\item
  zarówno obrazy podzielone na dwa, jak i trzy kategorie pokrycia terenu
  charakteryzują się podobnym rozkładem odpowiedzi
\item
  korelacja między miarami niepodobieństwa a odpowiedziami respondentów
  jest silniejsza dla par rastrów z dwoma kategoriami.
\end{itemize}

Przeanalizowane wyniki ankiety wraz z opisem przyjętej metodyki projektu
przedstawione zostały na posterze konferencyjnym pt.~„Porównanie metod
określania zmian struktury przestrzennej kategorii pokrycia terenu'',
autorstwa Błażeja Kościańskiego oraz dra hab. Jakuba Nowosada. Poster
został przedstawiony w trakcie sesji posterowej na konferencji
Geoinformacja: Nauka - Praktyka - Edukacja
(https://geoinformacja20uam.pl/), odbywającej się w dniach 1-3 grudnia
2022 roku na Wydziale Nauk Geograficznych i Geologicznych im. Adama
Mickiewicza w Poznaniu.

\bookmarksetup{startatroot}

\hypertarget{podsumowanie}{%
\chapter{Podsumowanie}\label{podsumowanie}}

\printbibliography[heading=bibintoc, title=Bibliografia]

\end{document}
